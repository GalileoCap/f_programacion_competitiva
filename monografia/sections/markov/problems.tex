\documentclass[../main.tex]{subfiles}

\begin{document}

\section{Problemas de Markov}

\subsection{\href{https://cses.fi/problemset/view/1726/}{CSES 1726 - Moving Robots}}

\paragraph{} En cada casillero de un tablero de ajedrez de \(8 \times 8\) se tiene un robot. En cada paso todos los robots se mueven aleatoriamente a uno de sus casilleros adyacentes. Se pueden tener varios robots en un mismo casillero.

Se busca calcular la cantidad esperada de casilleros vacíos después de \(k\) pasos.

Este problema se parece mucho al que hicimos antes, sólo que ahora en lugar de tener un sólo robot tenemos varios, y el tablero es más grande.

\subsubsection{Modelo}

\paragraph{} Primero vamos a querer simular a cada robot, para calcular su distribución después de \(k\) pasos.  Esto es simple: Calculamos la matriz de transición, la elevamos a \(k\), y luego la multiplicamos por cada vector inicial.

Una vez que tenemos las distribuciones para cada robot, podemos calcular la probabilidad de que no haya ningún robot en un casillero como el producto del complemento de las probabilidades de todos los robots. 
\begin{align*}
  P(\text{No haya ningún en } \langle r, c \rangle) &= P(\bigcap_{0 \leq i < 64} \text{El robot } i \text{ no esté en } \langle r, c \rangle) \\
                                                    &= \prod_{0 \leq i < 64}P(\text{El robot } i \text{ no esté en } \langle r, c \rangle) \\
                                                    &= \prod_{0 \leq i < 64}(1 - P(\text{El robot } i \text{ sí esté en } \langle r, c \rangle)) \\
                                                    &= \prod_{0 \leq i < 64}(1 - (M^{k}v_{i,0})_{8r + c}) \\
\end{align*}

Y ahora, teniendo la probabilidad de que no haya ningún robot en cada casillero. Tenemos que la cantidad esperada de casilleros vacíos es la suma de estas probabilidades.  
\begin{gather*}
  E(\text{Vacíos después de \(k\) pasos}) = \sum_{\langle r, c \rangle}1 \cdot P(\text{No haya ningún en } \langle r, c \rangle)
\end{gather*}

\subsubsection{Implementación}

\paragraph{} Para probarla rápidamente la primera implementación la hice en \texttt{Python}.

\lstinputlisting[language=Python]{cses_1726_moving_robots_v1.py}

\paragraph{} Esa implementación claramente no es lo suficientemente eficiente, por lo que proseguí a implementarlo en \texttt{C++}.

En esta segunda implementación en lugar de armar la matriz de transición y multiplicar la por el vector de estado que sería costoso, me aprovecho de que cada posición sólo se ve afectada por las cuatro posiciones a su alrededor y en \textit{stepRobot} simplifico la cuenta del producto.

Es importante resaltar que esta técnica devuelve el mismo resultado que el producto matricial ya que en efecto estoy haciendo lo mismo sólo que salteándome muchos productos por cero. Esta simplificación es muy útil en este estilo de ejercicios.

\lstinputlisting{cses_1726_moving_robots_v2.cc}

\paragraph{} Esa implementación ya era suficientemente eficiente. Pero es interesante ver que \(1 \leq k \leq 100\) y ese es el único input, por lo que al haber pocos casos, se pueden precalcular .

El siguiente script produce un vector de \texttt{C++} con los resultados ya calculados para \(0 \leq k \leq 100\) y luego el programa se simplifica (y optimiza) a sólo obtener el resultado pedido del vector.

\lstinputlisting{cses_1726_moving_robots_cheat.py}
\lstinputlisting{cses_1726_moving_robots_v3.cc}

\subsection{\href{https://www.codechef.com/ICPCIN19/problems/SHUFGRID}{CodeChef - Grid Shuffle}}

\paragraph{} Se tiene una grilla cuadrada de \(N \times N\). En cada paso se elije a una fila o una columna al azar, y los elementos de esta son reordenados al azar.

Se quiere saber la probabilidad de que un elemento cualquiera esté en su posición original después de \(k\) pasos.

El resultado se puede expresar como una fracción irreducible \(\frac{p}{q}\) donde \(q \perp 998244353\). Por lo que nos piden devolver \(p \cdot q^{-1} \text{ mod } 998244353\) con \(q^{-1}\) es el inverso multiplicativo de \(q\) módulo 998244353.

\subsubsection{Modelo}
\paragraph{} Primero vamos a solucionar el problema de calcular la probabilidad pedida, sin preocuparnos por el módulo.

Para modelarlo podemos resaltar que en todo momento hay cuatro estados posibles:
\begin{enumerate}
  \item El elemento sigue en su posición original (\Grrcc)
  \item El elemento está en la fila correcta pero en la columna incorrecta (\Grrcm)
  \item El elemento está en la fila incorrecta pero en la columna correcta (\Grmcc)
  \item El elemento está en la fila incorrecta y en la columna incorrecta (\Grmcm)
\end{enumerate}

Con estos cuatro estados posibles, ahora querríamos calcular la probabilidad de que se mueva entre ellos.

\paragraph{} Vamos armar la matriz de transición viendo el grafo de transición:

\paragraph{} Primero de \(\Grrcc\): Para movernos hacia \(\Grrcm\) en este paso tendríamos que elegir alterar una fila (\(\frac{1}{2}\) chances), alterar la fila en la que se encuentra el elemento (\(\frac{1}{n}\) chances), y que caiga en cualquiera de las otras posiciones de la fila (\(\frac{n-1}{n}\) chances), por lo que de \(\Grrcc\) a \(\Grrcm\) hay \(\beta = \frac{1}{2} \cdot \frac{1}{n} \cdot \frac{n-1}{n} = \GBETA\) chances. Es similar para \(\Grmcc\), y en el resto de los casos (se modifica otra fila/columna o el elemento cae en su lugar original) nos quedamos en el lugar correcto con probabilidad \(\alpha = 1 - 2\beta\).

\begin{figure}[H]
  \centering
  \begin{tikzpicture}
    \GVertex{\Grrcc}{0}{4}{0}
    \GVertex{\Grrcm}{4}{4}{1}
    \GVertex{\Grmcc}{0}{0}{2}
    \GVertex{\Grmcm}{4}{0}{3}

    \GLoop{$\alpha$}{0}{WE}
    \GBEdge{$\beta$}{0}{1}
    \GBEdge{$\beta$}{0}{2}
  \end{tikzpicture}
\end{figure}

\paragraph{} Luego desde \(\Grrcm\):
\begin{itemize}
  \item Para volver a \(\Grrcc\) querríamos elegir reordenar una fila (\(\frac{1}{2}\)), reordenar esta (\(\frac{1}{n}\)), y que caiga en la posición correcta (\(\frac{1}{n}\)), por lo que de \(\Grrcm\) a \(\Grrcc\) se tiene \(\gamma = \frac{1}{2} \cdot \frac{1}{n} \cdot \frac{1}{n} = \GGAMMA\) probabilidad.
  \item Luego para ir a \(\Grmcm\) ocurre lo mismo que de \(\Grrcc\) a \(\Grrcm\), así que tiene \(\beta\) chances de pasar.
  \item Y cualquier otro caso nos deja en el mismo estado: \(\delta = 1 - (\gamma + \beta)\).
\end{itemize}

Y similar para \(\Grmcc\).

\begin{figure}[H]
  \centering
  \begin{tikzpicture}
    \GVertex{\Grrcc}{0}{4}{0}
    \GVertex{\Grrcm}{4}{4}{1}
    \GVertex{\Grmcc}{0}{0}{2}
    \GVertex{\Grmcm}{4}{0}{3}

    \GLoop{$\alpha$}{0}{WE}
    \GBEdge{$\beta$}{0}{1}
    \GBEdge{$\beta$}{0}{2}

    \GLoop{$\delta$}{1}{EA}
    \GBEdge{$\gamma$}{1}{0}
    \GBEdge{$\beta$}{1}{3}

    \GLoop{$\delta$}{2}{WE}
    \GBEdge{$\gamma$}{2}{0}
    \GBEdge{$\beta$}{2}{3}
  \end{tikzpicture}
\end{figure}

\paragraph{} Y finalmente desde \(\Grmcm\):
\begin{itemize}
  \item Para volver a \(\Grrcm\) o \(\Grmcc\) tiene \(\gamma\) chances.
  \item Y para quedarse el resto de los casos: \(\epsilon = 1 - 2\gamma\).
\end{itemize}

\begin{figure}[H]
 \centering
 \begin{tikzpicture}
   \GVertex{\Grrcc}{0}{4}{0}
   \GVertex{\Grrcm}{4}{4}{1}
   \GVertex{\Grmcc}{0}{0}{2}
   \GVertex{\Grmcm}{4}{0}{3}

   \GLoop{$\alpha$}{0}{WE}
   \GBEdge{$\beta$}{0}{1}
   \GBEdge{$\beta$}{0}{2}

   \GLoop{$\delta$}{1}{EA}
   \GBEdge{$\gamma$}{1}{0}
   \GBEdge{$\beta$}{1}{3}

   \GLoop{$\delta$}{2}{WE}
   \GBEdge{$\gamma$}{2}{0}
   \GBEdge{$\beta$}{2}{3}

   \GLoop{$\epsilon$}{3}{EA}
   \GBEdge{$\gamma$}{3}{1}
   \GBEdge{$\gamma$}{3}{2}
 \end{tikzpicture}
\end{figure}

\paragraph{} Esto nos permite armar la matriz de transición:
\begin{gather*}
  M = \begin{pmatrix}
    \alpha & \beta & \beta & 0 \\
    \gamma & \delta & 0 & \beta \\
    \gamma & 0 & \delta & \beta \\
    0 & \gamma & \gamma & \epsilon
  \end{pmatrix}
\end{gather*}

Y un vector de estado de la forma:
\begin{gather*}
  v = \begin{pmatrix}
    P(\Grrcc) \\ P(\Grrcm) \\ P(\Grmcc) \\ P(\Grmcm)
  \end{pmatrix} \Rightarrow v_{0} = \begin{pmatrix}
    1 \\ 0 \\ 0 \\ 0
  \end{pmatrix}
\end{gather*}

\paragraph{} Y ahora, \((M^{k}v_{0})_{0} = (v_{k})_{0}\) es la probabilidad deseada.

Pero este no es el resultado pedido. Para conseguirlo, tendríamos que cambiar los elementos de la matriz de transición para que ya vayan haciendo los cálculos modulares.  En específico (tomando \(x^{-1}\) como el inverso multiplicativo de \(x\) módulo 998244353):
\begin{itemize}
  \item \(\beta = (n-1) \cdot (2n^{2})^{-1}\)
  \item \(\gamma = (2n^{2})^{-1}\)
\end{itemize}
 Y luego el resultado sería \((v_{k})_{0} \text{ mod } 998244353\).

\subsubsection{Implementación}

\paragraph{} La primera implementación la hice en \texttt{Python}, esta vez no usando librerías para poder entregarlo.

\lstinputlisting[language=Python]{codechef_grid_shuffle_v1.py}

\paragraph{} Esa implementación \textbf{sí} pasaba en tiempo, pero igualmente se puede hacer más rápido en \texttt{C++}.

Aunque la implementación se complica ya que es necesario implementar el obtener el inverso.

\lstinputlisting{codechef_grid_shuffle_v2.cc}

\end{document}
