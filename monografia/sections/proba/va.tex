\documentclass[../main.tex]{subfiles}

\begin{document}

\subsection{Variables aleatorias}

\paragraph{} Un uso común de la probabilidad es al realizar experimentos donde se mide un evento aleatorio. En estos casos es útil pensar en una función que represente a los resultados posibles. A esta función la nombramos \textbf{variable aleatoria (VA.)}. \textit{Variable} porque toma distintos valores y \textit{aleatoria} porque el valor observado no puede ser predicho antes de la realización del experimento, aunque sí se sabe cuales son sus posibles valores.

Dado que el valor de una VA. es determinado por el resultado de un experimento, podemos asignarle probabilidades a los posibles valores de la variable.

Como ejemplo, al arrojar dos veces un dado bien equilibrado. El espacio asociado es:
\begin{gather*}
  S = \{(x_{1}, x_{2}) / x_{1}, x_{2} \in \{1, 2, 3, 4, 5, 6\}\}
\end{gather*}

Y para este experimento tenemos (entre otras) las siguientes VA. asociadas:
\begin{itemize}
  \item \(X:\) ''número de caras pares''.
  \item \(Y\): ''el máximo''.
  \item \(Z\): ''la suma''.
\end{itemize}

\paragraph{} Formalmente, dado un espacio \(S\) definimos una variable aleatoria \(X\) como una función tal que:
\begin{gather*}
  X : S \rightarrow \mathbb{R}
\end{gather*}

Con los ejemplos anteriores tendríamos:
\begin{align*}
  X((2, 5)) &= 1 & Y((2, 5)) &= 5 & Z((2, 5)) &= 7 \\
  X((1, 3)) &= 0 & Y((1, 3)) &= 3 & Z((1, 3)) &= 4 \\
  X((2, 2)) &= 2 & Y((2, 2)) &= 2 & Z((2, 2)) &= 4 \\
\end{align*}

\paragraph{} Ahora entonces definimos la \textbf{función de probabilidad puntual} de la VA. \(X\) para todo \(x \in \mathbb{R}_{X}\), donde \(\mathbb{R}_{X}\) es el rango de \(X\) como:
\begin{gather*}
  P(X = x) = P(\{s \in S / X(s) = x\})
\end{gather*}

Que extiende la idea de probabilidad sobre eventos para que aplique sobre variables aleatorias donde el evento sería el caso en que \(X\) vale \(x\).

Es importante resaltar la siguiente propiedad derivada de que seguimos hablando sobre la frecuencia de eventos dentro de un espacio de posibilidades:
\begin{align*}
  \sum_{x \in \mathbb{R}_{X}}P(X = x) = 1
\end{align*}

\paragraph{} También definimos la \textbf{función de distribución acumulada} de una VA. \(X\) como la probabilidad de que \(X\) sea menor o igual que un valor \(x\), formalmente:
\begin{align*}
  P(X \leq x) = \sum_{y \leq x, y \in \mathbb{R}_{X}}P(X = y) && \forall x \in \mathbb{R}
\end{align*}

\subsubsection{Esperanza y Varianza}

\paragraph{} Sobre una VA. \(X\) definimos la \textbf{esperanza} \(E(X)\) como el valor que tiene en promedio que se calcula de la siguiente manera:
\begin{gather*}
  E(X) = \mu_{X} = \sum_{x \in \mathbb{R}_{X}}x \cdot P(X = x)
\end{gather*}
Siempre que \(\sum_{x \in \mathbb{R}_{X}}\abs{x} \cdot P(X = x) < \infty\), ya que si la serie de los valores absolutos diverge, la esperanza no puede definirse y decimos que no existe.

Por ejemplo, al tirar un dado:
\begin{gather*}
  E(X) = 1 \cdot \frac{1}{6} + 2 \cdot \frac{1}{6} + 3 \cdot \frac{1}{6} + 4 \cdot \frac{1}{6} + 5 \cdot \frac{1}{6} + 6 \cdot \frac{1}{6} = \frac{7}{2}
\end{gather*}

Es importante resaltar que a pesar del nombre y de que se hable de ''el valor esperado'', el valor de la esperanza no necesariamente es un valor posible de \(X\). Sino que es el valor que se espera conseguir al promediar muchas mediciones de \(X\). Esto se ve con la \textbf{Ley de los Grandes Números}.

% TODO: Histograma, con esperanza marcada. Para dar pié a la varianza

\subsubsection{Distribuciones}

\end{document}
