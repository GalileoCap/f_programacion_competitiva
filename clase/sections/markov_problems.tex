\documentclass[../main.tex]{subfiles}

\begin{document}

\newcommand{\SECTIOND}{Problemas de Markov}
\section{\SECTIOND}

\newcommand{\EJD}{\href{https://cses.fi/problemset/view/1726/}{CSES 1726 - Moving Robots}}
\subsection{CSES 1726 - Moving Robots}

\begin{frame}
  \frametitle{\SECTIOND}
  \framesubtitle{\EJD}

  En cada casillero de un tablero de ajedrez de \(8 \times 8\) se tiene un robot. En cada paso todos los robots se mueven aleatoriamente a uno de sus casilleros adyacentes. Se pueden tener varios robots en un mismo casillero. \\
  Se busca calcular la cantidad esperada de casilleros vacíos después de \(k\) pasos. \pause \\
  Este problema se parece mucho al que hicimos antes, sólo que ahora en lugar de tener un sólo robot tenemos varios, y el tablero es más grande.
\end{frame}

\begin{frame}
  \frametitle{\SECTIOND}
  \framesubtitle{\EJD, para cada casillero}

  Primero vamos a querer simular a cada robot, para calcular su distribución después de \(k\) pasos. \pause Esto es simple: Calculamos la matriz de transición, la elevamos a \(k\), y luego la multiplicamos por cada vector inicial. \pause \\
  Una vez que tenemos las distribuciones para cada robot, podemos calcular la probabilidad de que no haya ningún robot en un casillero como el producto del complemento de las probabilidades de todos los robots. \pause
  \begin{align*}
    P(\text{No haya ningún en } \langle r, c \rangle) &= P(\bigcap_{0 \leq i < 64} \text{El robot } i \text{ no esté en } \langle r, c \rangle) \\
                                                      &= \prod_{0 \leq i < 64}P(\text{El robot } i \text{ no esté en } \langle r, c \rangle) \\
                                                      &= \prod_{0 \leq i < 64}(1 - P(\text{El robot } i \text{ sí esté en } \langle r, c \rangle)) \\
                                                      &= \prod_{0 \leq i < 64}(1 - (M^{k}v_{i,0})_{8r + c}) \\
  \end{align*}
\end{frame}

\begin{frame}
  \frametitle{\SECTIOND}
  \framesubtitle{\EJD, solución}

  Y ahora, teniendo la probabilidad de que no haya ningún robot en cada casillero. Tenemos que la cantidad esperada de casilleros vacíos es la suma de estas probabilidades. \pause 
  \begin{gather*}
    E(\text{Vacíos después de \(k\) pasos}) = \sum_{\langle r, c \rangle}1 \cdot P(\text{No haya ningún en } \langle r, c \rangle)
  \end{gather*}
\end{frame}

\newcommand{\EJE}{\href{https://www.codechef.com/ICPCIN19/problems/SHUFGRID}{CodeChef - Grid Shuffle}}
\subsection{CodeChef - Grid Shuffle}

\begin{frame}
  \frametitle{\SECTIOND}
  \framesubtitle{\EJE}

  Se tiene una grilla cuadrada de \(N \times N\). En cada paso se elije a una fila o una columna al azar, y los elementos de esta son reordenados al azar. \\
  Se quiere saber la probabilidad de que un elemento cualquiera esté en su posición original después de \(k\) pasos. \pause \\
  El resultado se puede expresar como una fracción irreducible \(\frac{p}{q}\) donde \(q \perp 998244353\). Por lo que nos piden devolver \(p \cdot q^{-1} \text{ mod } 998244353\) con \(q^{-1}\) es el inverso multiplicativo de \(q\) módulo 998244353.
\end{frame}

\begin{frame}
  \frametitle{\SECTIOND}
  \framesubtitle{\EJE, modelado}

  Primero vamos a solucionar el problema de calcular la probabilidad pedida, sin preocuparnos por el módulo. \pause \\
  Para modelarlo podemos resaltar que en todo momento hay cuatro estados posibles:
  \begin{enumerate}
    \item<2-> El elemento sigue en su posición original (\Grrcc)
    \item<3-> El elemento está en la fila correcta pero en la columna incorrecta (\Grrcm)
    \item<4-> El elemento está en la fila incorrecta pero en la columna correcta (\Grmcc)
    \item<5-> El elemento está en la fila incorrecta y en la columna incorrecta (\Grmcm)
  \end{enumerate}
  \pause \pause \pause \pause Con estos cuatro estados posibles, ahora querríamos calcular la probabilidad de que se mueva entre ellos.
\end{frame}

\begin{frame}
  \frametitle{\SECTIOND}
  \framesubtitle{\EJE, grafo de transición}

  Con: \(\beta = \GBETA, \alpha = 1 - 2\beta\)

  \begin{figure}[H]
    \centering
    \begin{tikzpicture}
      \GVertex{\Grrcc}{0}{4}{0}
      \GVertex{\Grrcm}{4}{4}{1}
      \GVertex{\Grmcc}{0}{0}{2}
      \GVertex{\Grmcm}{4}{0}{3}
    \end{tikzpicture}
  \end{figure}
\end{frame}

\begin{frame}
  \frametitle{\SECTIOND}
  \framesubtitle{\EJE, grafo de transición}

  Con: \(\beta = \GBETA, \alpha = 1 - 2\beta\)

  \begin{figure}[H]
    \centering
    \begin{tikzpicture}
      \GVertex{\Grrcc}{0}{4}{0}
      \GVertex{\Grrcm}{4}{4}{1}
      \GVertex{\Grmcc}{0}{0}{2}
      \GVertex{\Grmcm}{4}{0}{3}

      \GLoop{$\alpha$}{0}{WE}
      \GBEdge{$\beta$}{0}{1}
      \GBEdge{$\beta$}{0}{2}
    \end{tikzpicture}
  \end{figure}
\end{frame}

\begin{frame}
  \frametitle{\SECTIOND}
  \framesubtitle{\EJE, grafo de transición}

  Con: \(\beta = \GBETA, \alpha = 1 - 2\beta, \gamma = \GGAMMA, \delta = 1 - (\gamma + \beta)\)

  \begin{figure}[H]
    \centering
    \begin{tikzpicture}
      \GVertex{\Grrcc}{0}{4}{0}
      \GVertex{\Grrcm}{4}{4}{1}
      \GVertex{\Grmcc}{0}{0}{2}
      \GVertex{\Grmcm}{4}{0}{3}

      \GLoop{$\alpha$}{0}{WE}
      \GBEdge{$\beta$}{0}{1}
      \GBEdge{$\beta$}{0}{2}

      \GLoop{$\delta$}{1}{EA}
      \GBEdge{$\gamma$}{1}{0}
      \GBEdge{$\beta$}{1}{3}
    \end{tikzpicture}
  \end{figure}
\end{frame}

\begin{frame}
  \frametitle{\SECTIOND}
  \framesubtitle{\EJE, grafo de transición}

  Con: \(\beta = \GBETA, \alpha = 1 - 2\beta, \gamma = \GGAMMA, \delta = 1 - (\gamma + \beta)\)

  \begin{figure}[H]
    \centering
    \begin{tikzpicture}
      \GVertex{\Grrcc}{0}{4}{0}
      \GVertex{\Grrcm}{4}{4}{1}
      \GVertex{\Grmcc}{0}{0}{2}
      \GVertex{\Grmcm}{4}{0}{3}

      \GLoop{$\alpha$}{0}{WE}
      \GBEdge{$\beta$}{0}{1}
      \GBEdge{$\beta$}{0}{2}

      \GLoop{$\delta$}{1}{EA}
      \GBEdge{$\gamma$}{1}{0}
      \GBEdge{$\beta$}{1}{3}

      \GLoop{$\delta$}{2}{WE}
      \GBEdge{$\gamma$}{2}{0}
      \GBEdge{$\beta$}{2}{3}
    \end{tikzpicture}
  \end{figure}
\end{frame}

\begin{frame}
  \frametitle{\SECTIOND}
  \framesubtitle{\EJE, grafo de transición}

  Con: \(\beta = \GBETA, \alpha = 1 - 2\beta, \gamma = \GGAMMA, \delta = 1 - (\gamma + \beta), \epsilon = 1 - 2\gamma\)

  \begin{figure}[H]
    \centering
    \begin{tikzpicture}
      \GVertex{\Grrcc}{0}{4}{0}
      \GVertex{\Grrcm}{4}{4}{1}
      \GVertex{\Grmcc}{0}{0}{2}
      \GVertex{\Grmcm}{4}{0}{3}

      \GLoop{$\alpha$}{0}{WE}
      \GBEdge{$\beta$}{0}{1}
      \GBEdge{$\beta$}{0}{2}

      \GLoop{$\delta$}{1}{EA}
      \GBEdge{$\gamma$}{1}{0}
      \GBEdge{$\beta$}{1}{3}

      \GLoop{$\delta$}{2}{WE}
      \GBEdge{$\gamma$}{2}{0}
      \GBEdge{$\beta$}{2}{3}

      \GLoop{$\epsilon$}{3}{EA}
      \GBEdge{$\gamma$}{3}{1}
      \GBEdge{$\gamma$}{3}{2}
    \end{tikzpicture}
  \end{figure}
\end{frame}

\begin{frame}
  \frametitle{\SECTIOND}
  \framesubtitle{\EJE, matriz de transición}

  Esto nos permite armar la matriz de transición:
  \begin{gather*}
    M = \begin{pmatrix}
      \alpha & \beta & \beta & 0 \\
      \gamma & \delta & 0 & \beta \\
      \gamma & 0 & \delta & \beta \\
      0 & \gamma & \gamma & \epsilon
    \end{pmatrix}
  \end{gather*}
  Con: \(\beta = \GBETA, \alpha = 1 - 2\beta, \gamma = \GGAMMA, \delta = 1 - (\gamma + \beta), \epsilon = 1 - 2\gamma\). \pause \\
  Y un vector de estado de la forma:
  \begin{gather*}
    v = \begin{pmatrix}
      P(\Grrcc) \\ P(\Grrcm) \\ P(\Grmcc) \\ P(\Grmcm)
    \end{pmatrix} \Rightarrow v_{0} = \begin{pmatrix}
      1 \\ 0 \\ 0 \\ 0
    \end{pmatrix}
  \end{gather*}
\end{frame}

\begin{frame}
  \frametitle{\SECTIOND}
  \framesubtitle{\EJE, resultado}

  Y ahora, \((M^{k}v_{0})_{0} = (v_{k})_{0}\) es la probabilidad deseada. \pause \\
  Pero esta no es el resultado pedido. \pause Para conseguirlo, tendríamos que cambiar los elementos de la matriz de transición para que ya vayan haciendo los cálculos modulares. \pause En específico (tomando \(x^{-1}\) como el inverso multiplicativo de \(x\) módulo 998244353):
  \begin{itemize}
    \item \(\beta = (n-1) \cdot (2n^{2})^{-1}\)
    \item \(\gamma = (2n^{2})^{-1}\)
  \end{itemize}
  \pause Y luego el resultado sería \((v_{k})_{0} \text{ mod } 998244353\).
\end{frame}

\end{document}
